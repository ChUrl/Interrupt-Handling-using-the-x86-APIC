\makeglossaries

%%%%%%%%%%%%%%%%%%%%%%%%%%%%%%%%%%%%%%%%%%%%%%%%%%%%%%%%%%%%%%%%%%%%%%%
% Acronyms
%%%%%%%%%%%%%%%%%%%%%%%%%%%%%%%%%%%%%%%%%%%%%%%%%%%%%%%%%%%%%%%%%%%%%%%

\newacronym{acpi}{ACPI}{\gls{advanced configuration and power interface}}
\newacronym{aml}{AML}{acpi machine language}
\newacronym{ap}{AP}{\gls{application processor}}
\newacronym{apr}{APR}{arbitration priority register}
\newacronym{apic}{APIC}{advanced programmable interrupt controller}
\newacronym{bios}{BIOS}{basic input/output system}
\newacronym{bsp}{BSP}{\gls{bootstrap processor}}
\newacronym{cpuid}{CPUID}{cpu identification}
\newacronym{eoi}{EOI}{\gls{end-of-interrupt}}
\newacronym{esr}{ESR}{error status register}
\newacronym{gdt}{GDT}{global descriptor table}
\newacronym{gdtr}{GDTR}{global descriptor table register}
\newacronym{gsi}{GSI}{\gls{global system interrupt}}
\newacronym{ich}{ICH}{intel input/output controller hub}
\newacronym{icr}{ICR}{\gls{interrupt command register}}
\newacronym{idt}{IDT}{interrupt descriptor table}
\newacronym{idtr}{IDTR}{interrupt descriptor table register}
\newacronym{imcr}{IMCR}{\gls{interrupt mode control register}}
\newacronym{imr}{IMR}{interrupt mask register}
\newacronym{inti}{INTI}{interrupt input}
\newacronym{ipi}{IPI}{inter-processor interrupt}
\newacronym{irq}{IRQ}{interrupt request}
\newacronym{irr}{IRR}{\gls{interrupt request register}}
\newacronym{isa}{ISA}{industry standard architecture}
\newacronym{isr}{ISR}{\gls{in-service register}}
\newacronym{lvt}{LVT}{\gls{local vector table}}
\newacronym{madt}{MADT}{multiple apic descriptor table}
\newacronym{mmio}{MMIO}{memory mapped input/output}
\newacronym{msr}{MSR}{model specific register}
\newacronym{msi}{MSI}{\gls{message-signaled interrupt}}
\newacronym{nmi}{NMI}{\gls{non-maskable interrupt}}
\newacronym{pci}{PCI}{peripheral component interconnect}
\newacronym{pic}{PIC}{programmable interrupt controller}
\newacronym{pit}{PIT}{\gls{programmable interval timer}}
\newacronym{redtbl}{REDTBL}{redirection table}
\newacronym{sipi}{SIPI}{startup inter-processor interrupt}
\newacronym{smi}{SMI}{system management interrupt}
\newacronym{smp}{SMP}{\gls{symmetric multiprocessing}}
\newacronym{svr}{SVR}{\gls{spurious interrupt vector register}}
\newacronym{tmr}{TMR}{trigger-mode register}
\newacronym{tpr}{TPR}{\gls{task-priority register}}
\newacronym{tss}{TSS}{task state segment}
\newacronym{uefi}{UEFI}{unified extensible firmware interface}

%%%%%%%%%%%%%%%%%%%%%%%%%%%%%%%%%%%%%%%%%%%%%%%%%%%%%%%%%%%%%%%%%%%%%%%
% Glossary Entries
%%%%%%%%%%%%%%%%%%%%%%%%%%%%%%%%%%%%%%%%%%%%%%%%%%%%%%%%%%%%%%%%%%%%%%%

% \newglossaryentry{}{
%   name={},
%   description={}
% }

\newglossaryentry{advanced configuration and power interface}{
  name={advanced configuration and power interface},
  description={a standard for operating systems to allow software to query information about the system hardware configuration}
}
\newglossaryentry{apic timer}{
  name={APIC timer},
  description={a hardware timer that can trigger periodic interrupts by using a counter, integrated into the local APIC}
}
\newglossaryentry{application processor}{
  name={application processor},
  description={a processor inside an SMP system, e.g. a CPU core}
}
\newglossaryentry{bootstrap processor}{
  name={bootstrap processor},
  description={the application processor used to boot an SMP system}
}
\newglossaryentry{cascaded interrupt}{
  name={cascaded interrupt},
  description={an interrupt, serviced during another interrupt handler}
}
\newglossaryentry{end-of-interrupt}{
  name={end-of-interrupt},
  description={a notification to an interrupt controller that an interrupt has been handled}
}
\newglossaryentry{external interrupt}{
  name={external interrupt},
  description={an interrupt from an external hardware device}
}
\newglossaryentry{global system interrupt}{
  name={global system interrupt},
  description={an abstraction used by ACPI to decouple interrupts from hardware interrupt lines}
}
\newglossaryentry{ia32 apic base msr}{
  name={IA32\textunderscore{}APIC\textunderscore{}BASE MSR},
  description={an x86 architectural MSR that contains the local APIC's physical MMIO address and the xApic global enable/disable flag}
}
\newglossaryentry{io apic}{
  name={I/O APIC},
  description={a part of the APIC architecture inside the chipset, responsible for receiving external interrupts}
}
\newglossaryentry{init ipi}{
  name={INIT IPI},
  description={an interprocessor interrupt sent from the BSP to the APs, to begin the AP initialization process}
}
\newglossaryentry{inter-processor interrupt}{
  name={inter-processor interrupt},
  description={an interrupt sent between CPU cores}
}
\newglossaryentry{interrupt}{
  name={interrupt},
  description={a request for the CPU to handle an event}
}
\newglossaryentry{interrupt command register}{
  name={interrupt command register},
  description={a register of the local APIC used to issue interprocessor interrupts}
}
\newglossaryentry{interrupt controller}{
  name={interrupt controller},
  description={a hardware component designated to receive interrupts and forward them to the CPU}
}
\newglossaryentry{interrupt handler}{
  name={interrupt handler},
  description={a function designated to handle a specific interrupt}
}
\newglossaryentry{interrupt mode control register}{
  name={interrupt mode control register},
  description={a register present in some systems to choose the physically connected interrupt controller}
}
\newglossaryentry{interrupt priority}{
  name={interrupt priority},
  description={the interrupt priority decides the order, in which multiple, simultaneously arriving interrupts, are forwarded to the CPU}
}
\newglossaryentry{interrupt vector}{
  name={interrupt vector},
  description={a slot of an interrupt handler in the IDT}
}
\newglossaryentry{interrupt request register}{
  name={interrupt request register},
  description={a register, part of the PIC and local APIC, which keeps track of received interrupts}
}
\newglossaryentry{in-service register}{
  name={in-service register},
  description={a register, part of the PIC and local APIC, which keeps track of interrupts that are being serviced}
}
\newglossaryentry{irq override}{
  name={IRQ override},
  description={information from ACPI, that describes how interrupt lines correspond to global system interrupts}
}
\newglossaryentry{local apic}{
  name={local APIC},
  description={a part of the APIC architecture inside a CPU core, responsible for receiving local interrupts and communication with the I/O APIC}
}
\newglossaryentry{local interrupt}{
  name={local interrupt},
  description={an CPU internal interrupt handled by the local APIC, like the APIC timer interrupt}
}
\newglossaryentry{local vector table}{
  name={local vector table},
  description={a set of registers, part of the local APIC, that configure how local interrupts are handled}
}
\newglossaryentry{masking}{
  name={masking},
  description={marking an interrupt as ignored}
}
\newglossaryentry{memory-mapped io}{
  name={memory-mapped I/O},
  description={a way to access a devices registers by mapping them to addresses in the main memory}
}
\newglossaryentry{message-signaled interrupt}{
  name={message-signaled interrupt},
  description={an interrupt sent in-band over a PCI-bus}
}
\newglossaryentry{non-maskable interrupt}{
  name={non-maskable interrupt},
  description={a system critical interrupt that cannot be masked}
}
\newglossaryentry{pcat pic architecture}{
  name={PC/AT PIC architecture},
  description={an interrupt controller configuration using two cascaded PICs for a total of 15 interrupt lines}
}
\newglossaryentry{pic mode}{
  name={PIC mode},
  description={only use the PIC for interrupt handling, like the PC/AT}
}
\newglossaryentry{pin polarity}{
  name={pin polarity},
  description={describes if a signal is represented by either a high or low level change or theshold}
}
\newglossaryentry{programmable interval timer}{
  name={programmable interval timer},
  description={a hardware timer that can trigger periodic interrupts by using a counter}
}
\newglossaryentry{protected mode}{
  name={protected mode},
  description={the 32-bit operating mode of x86 CPUs which includes security features such as paging}
}
\newglossaryentry{real mode}{
  name={real mode},
  description={an operating mode of x86 CPUs with 16-bit segmented memory addressing, resulting in 1 MB of usable memory with 20-bit address width}
}
\newglossaryentry{redirection table}{
  name={redirection table},
  description={a set of registers, part of the I/O APIC, that configure how external interrupts are handled}
}
\newglossaryentry{spurious interrupt}{
  name={spurious interrupt},
  description={an interrupt that is triggered in case an original interrupt is no longer valid on delivery}
}
\newglossaryentry{spurious interrupt vector register}{
  name={spurious interrupt vector register},
  description={a register of the local APIC which contains the APIC software enable flag and the spurious interrupt vector}
}
\newglossaryentry{startup ipi}{
  name={STARTUP IPI},
  description={an interprocessor interrupt sent from the BSP to the APs, to load the AP startup routine and finish AP initialization}
}
\newglossaryentry{symmetric io mode}{
  name={symmetric I/O mode},
  description={use the I/O APIC in combination with the local APIC for interrupt handling in multiprocessor systems}
}
\newglossaryentry{symmetric multiprocessing}{
  name={symmetric multiprocessing},
  description={a computer architecture where multiple CPUs operate on a shared main memory}
}
\newglossaryentry{task-priority register}{
  name={task-priority register},
  description={a register of the local APIC which determines interrupt handling order and priority theshold}
}
\newglossaryentry{trigger mode}{
  name={trigger mode},
  description={describes if a signal is represented by either a change in level or a level theshold}
}
\newglossaryentry{virtual wire mode}{
  name={virtual wire mode},
  description={use the local APIC in combination with the PIC as external interrupt controller}
}
\newglossaryentry{xapic}{
  name={xApic},
  description={a revision of the APIC architecture, register access is handled through MMIO}
}
\newglossaryentry{x2apic}{
  name={x2Apic},
  description={a revision of the APIC architecture, register access is handled through MSRs}
}
